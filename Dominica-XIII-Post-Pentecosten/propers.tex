% This file defines the propert to be inserted into missalette.tex  In this way
% multiple feasts can be typeset very quickly.  missalette.tex should not normally
% need changing.  Note that this is not the most readable way to insert text
% into a LaTeX document, but it is the most powerful: the macros defined here
% are directly excecuted when building the document.

% For the title page
\newcommand{\feast}{Dominica XIII Post Pentecosten}
\newcommand{\masstype}{ Missa Cantata%
  % %
}
% 
\newcommand{\introit}{%
  Réspice, Dómine, in testaméntum tuum, et ánimas páuperum tuórum ne derelínquas in finem: exsúrge, Dómine, et iúdica causam tuam, et ne obliviscáris voces quæréntium te.
  Ut quid, Deus, reppulísti in finem: irátus est furor tuus super oves páscuæ tuæ?
}
\newcommand{\introitTranslation}{%
  Look to Your covenant, O Lord, forsake not forever the lives of Your afflicted ones. Arise, O Lord; defend Your cause; be not unmindful of the voices of those who ask You.
  Why, O God, have You cast us off forever? Why does Your anger smolder against the sheep of Your pasture?
}
% 
\newcommand{\collect}{%
  \l{%
    Omnípotens sempitérne Deus, da nobis fídei, spei et caritátis augméntum: et, ut mereámur asséqui quod promíttis, fac nos amáre quod prǽcipis.
  }
  \e{%
    Almighty, eternal God, grant us an increase of faith, hope and charity; and make us love what You command so that we may be made worthy to attain what You promise.
  }
  \per
}
\newcommand{\lesson}{%
  \l{%
    Fratres: Abrahæ dictæ sunt promissiónes, et sémini eius. Non dicit: Et semínibus, quasi in multis; sed quasi in uno: Et sémini tuo, qui est Christus. Hoc autem dico: testaméntum confirmátum a Deo, quæ post quadringéntos et trigínta annos facta est lex, non írritum facit ad evacuándam promissiónem. Nam si ex lege heréditas, iam non ex promissióne. Abrahæ autem per repromissiónem donávit Deus. Quid igitur lex? Propter transgressiónes pósita est, donec veníret semen, cui promíserat, ordináta per Angelos in manu mediatóris. Mediátor autem uníus non est: Deus autem unus est. Lex ergo advérsus promíssa Dei? Absit. Si enim data esset lex, quæ posset vivificáre, vere ex lege esset iustítia. Sed conclúsit Scriptúra ómnia sub peccáto, ut promíssio ex fide Iesu Christi darétur credéntibus.
  }
  \e{%
    Brethren: The promises were made to Abraham and to his offspring. He does not say, And to his offsprings, as of many; but as of one, And to your offspring, Who is Christ. Now I mean this: The Law which was made four hundred and thirty years later does not annul the covenant which was ratified by God, so as to make the promise void. For if the right to inherit be from the Law, it is no longer from a promise. But God gave it to Abraham by promise. What then was the Law? It was enacted on account of transgressions, being delivered by angels through a mediator, until the offspring should come to whom the promise was made. Now there is no intermediary where there is only one; but God is one. Is the Law then contrary to the promises of God? By no means. For if a law had been given that could give life, justice would truly be from the Law. But the Scriptures shut up all things under sin, that by the faith of Jesus Christ the promise might be given to those who believe.
  }
}

% 
\newcommand{\gradual}{%
  Réspice, Dómine, in testaméntum tuum: et ánimas páuperum tuórum ne obliviscáris in finem.
  Exsúrge, Dómine, et iúdica causam tuam: memor esto oppróbrii servórum tuórum. Allelúia, allelúia
  Dómine, refúgium factus es nobis a generatióne et progénie. Allelúia.
}
\newcommand{\gradualTranslation}{%
  Look to Your covenant, O Lord, be not unmindful of the lives of Your afflicted ones.
  Arise, O Lord; defend Your cause; remember the reproach of Your servants. Alleluia, alleluia.
  O Lord, You have been our refuge through all generations. Alleluia.
}
% 
% 
\newcommand{\gospel}{%
  \l{%
    In illo témpore: Dum iret Iesus in Ierúsalem, transíbat per médiam Samaríam et Galilǽam. Et cum ingrederétur quoddam castéllum, occurrérunt ei decem viri leprósi, qui stetérunt a longe; et levavérunt vocem dicéntes: Iesu præcéptor, miserére nostri. Quos ut vidit, dixit: Ite, osténdite vos sacerdótibus. Et factum est, dum irent, mundáti sunt. Unus autem ex illis, ut vidit quia mundátus est, regréssus est, cum magna voce magníficans Deum, et cecidit in fáciem ante pedes eius, grátias agens: et hic erat Samaritánus. Respóndens autem Iesus, dixit: Nonne decem mundáti sunt? et novem ubi sunt? Non est invéntus, qui redíret et daret glóriam Deo, nisi hic alienígena. Et ait illi: Surge, vade; quia fides tua te salvum fecit.
  }
  \e{%
    At that time, Jesus was going to Jerusalem, He was passing between Samaria and Galilee. And as He was entering a certain village, there He met ten lepers, who stood afar off and lifted up their voice, crying, Jesus, Master, have pity on us. And when He saw them He said, Go, show yourselves to the priests. And it came to pass as they were on their way, that they were made clean. But one of them, seeing that he was made clean, returned, with a loud voice glorifying God, and he fell on his face at His feet, giving thanks; and he was a Samaritan. But Jesus answered and said, Were not the ten made clean? But where are the nine? Has no one been found to return and give glory to God, except this foreigner? And He said to him, Arise, go your way, for your faith has saved you.
  }
}
\newcommand{\offertory}{%
  In te sperávi, Dómine; dixi: Tu es Deus meus, in mánibus tuis témpora mea.
}
\newcommand{\offertoryTranslation}{%
  My trust is in You, O Lord; I say, You are my God. In Your hands is my destiny.
}
\newcommand{\secret}{%
  \l{%
    Propitiáre, Dómine, pópulo tuo, propitiáre munéribus: ut, hac oblatióne placátus, et indulgéntiam nobis tríbuas et postuláta concedas.
  }
  \e{%
    Look with favor upon Your people, O Lord, look with favor upon their gifts; so that, appeased by this offering, You will grant us pardon and give us what we ask.
  }
  \per
}
\newcommand{\communion}{%
  Panem de cœlo dedísti nobis, Dómine, habéntem omne delectaméntum et omnem sapórem suavitátis.
}
\newcommand{\communionTranslation}{%
  You have given us, O Lord, bread from heaven, endowed with all delights and the sweetness of every taste.
}
\newcommand{\postcommunion}{%
  \l{%
    Sumptis, Dómine, cœléstibus sacraméntis: ad redemptiónis ætérnæ, quǽsumus, proficiámus augméntum.
  }
  \e{%
    Having received Your heavenly sacrament, O Lord, may we make progress, we beseech You, toward our everlasting salvation.
  }
  \per
}

% File paths: we don't use symlinks as (a) not all platforms support them, and
% (b) they don't fit nicely with the flow we're using.


\newcommand{\kyriePath}{../Ordinaries/masses/8/kyrie}

\newcommand{\gloriaPath}{../Ordinaries/masses/8/gloria}

\newcommand{\sanctusPath}{../Ordinaries/masses/8/sanctus}

\newcommand{\agnusPath}{../Ordinaries/masses/8/agnus}

\newcommand{\itePath}{../Ordinaries/masses/8/ite}


\newcommand{\creedPath}{../Ordinaries/credo/1/credo}


\newcommand{\amenPath}{../ToniCommunes/roman/amen}

\newcommand{\aspergesPath}{../ToniCommunes/roman/asperges}

\newcommand{\dominusVobiscumPath}{../ToniCommunes/roman/dominus-vobiscum}

\newcommand{\paxDominiPath}{../ToniCommunes/roman/pax-domini}

\newcommand{\prefacePath}{../ToniCommunes/roman/preface_standard}

\newcommand{\sedLiberaNosPath}{../ToniCommunes/roman/sed-libera-nos}

\newcommand{\sequentiPath}{../ToniCommunes/roman/sequenti}


\newcommand{\marianPath}{../MarianAntiphons/roman/salve-regina}
\input{../MarianAntiphons/salve-regina_resp}

%%% Local Variables:
%%% mode: latex
%%% TeX-master: "missalette"
%%% End:
