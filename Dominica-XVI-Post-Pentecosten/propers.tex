% This file defines the propert to be inserted into missalette.tex  In this way
% multiple feasts can be typeset very quickly.  missalette.tex should not normally
% need changing.  Note that this is not the most readable way to insert text
% into a LaTeX document, but it is the most powerful: the macros defined here
% are directly excecuted when building the document.

% For the title page
\newcommand{\feast}{Dominica XVI Post Pentecosten}
\newcommand{\masstype}{ Missa Cantata%
  % %
}
% 
\newcommand{\introit}{%
Miserére mihi, Dómine, quóniam ad te clamávi tota die: quia tu, Dómine, suávis ac mitis es, et copiósus in misericórdia ómnibus invocántibus te.
Inclína, Dómine, aurem tuam mihi, et exáudi me: quóniam inops, et pauper sum ego.
}
\newcommand{\introitTranslation}{%
Have pity on me, O Lord, for to You I call all the day; for You, O Lord, are good and forgiving, abounding in kindness to all who call upon You.
Incline Your ear, O Lord; answer me, for I am afflicted and poor.
}
% 
\newcommand{\collect}{%
  \l{%
Tua nos, quǽsumus, Dómine, grátia semper et prævéniat et sequátur: ac bonis opéribus iúgiter præstet esse inténtos.
  }
  \e{%
May Your grace, we beseech You, O Lord, ever go before us and follow us, and may it make us ever intent upon good works.
  }
  \per
}
\newcommand{\lesson}{%
  \l{%
Fratres: Obsecro vos, ne deficiátis in tribulatiónibus meis pro vobis: quæ est glória vestra. Huius rei grátia flecto génua mea ad Patrem Dómini nostri Iesu Christi, ex quo omnis patérnitas in cœlis et in terra nominátur, ut det vobis secúndum divítias glóriæ suæ, virtúte corroborári per Spíritum eius in interiórem hóminem, Christum habitáre per fidem in córdibus vestris: in caritáte radicáti et fundáti, ut póssitis comprehéndere cum ómnibus sanctis, quæ sit latitúdo et longitúdo et sublímitas et profúndum: scire étiam supereminéntem sciéntiæ caritátem Christi, ut impleámini in omnem plenitúdinem Dei. Ei autem, qui potens est ómnia fácere superabundánter, quam pétimus aut intellégimus, secúndum virtútem, quæ operátur in nobis: ipsi glória in Ecclésia et in Christo Iesu, in omnes generatiónes sǽculi sæculórum. Amen.
  }
  \e{%
Brethren: I pray you not to be disheartened at my tribulations for you, for they are your glory. For this reason I bend my knees to the Father of our Lord Jesus Christ, from Whom all fatherhood in heaven and on earth receives its name, that He may grant you from His glorious riches to be strengthened with power through His Spirit unto the progress of the inner man; and to have Christ dwelling through faith in your hearts: so that, being rooted and grounded in love, you may be able to comprehend with all the saints what is the breadth and length and height and depth, and to know Christ’s love which surpasses knowledge, in order that you may be filled unto all the fullness of God. Now, to Him Who is able to accomplish all things in a measure far beyond what we ask or conceive, in keeping with the power that is at work in us - to Him be glory in the Church and in Christ Jesus down through all the ages of time without end. Amen.
  }
}

% 
\newcommand{\gradual}{%
Timébunt gentes nomen tuum, Dómine, et omnes reges terræ glóriam tuam.
Quóniam ædificávit Dóminus Sion, et vidébitur in maiestáte sua. Allelúia, allelúia
Cantáte Dómino cánticum novum: quia mirabília fecit Dóminus. Allelúia.
}
\newcommand{\gradualTranslation}{%
The nations shall revere Your name, O Lord, and all the kings of the earth Your glory.
For the Lord has rebuilt Sion, and He shall appear in His glory. Alleluia, alleluia.
Sing to the Lord a new song, for the Lord has done wondrous deeds. Alleluia.
}
% 
\newcommand{\gospel}{%
  \l{%
In illo témpore: Cum intráret Iesus in domum cuiúsdam príncipis pharisæórum sábbato manducáre panem, et ipsi observábant eum. Et ecce, homo quidam hydrópicus erat ante illum. Et respóndens Iesus dixit ad legisperítos et pharisǽos, dicens: Si licet sábbato curáre? At illi tacuérunt. Ipse vero apprehénsum sanávit eum ac dimísit. Et respóndens ad illos, dixit: Cuius vestrum ásinus aut bos in púteum cadet, et non contínuo éxtrahet illum die sábbati? Et non póterant ad hæc respóndere illi. Dicebat autem et ad invitátos parábolam, inténdens, quómodo primos accúbitus elígerent, dicens ad illos: Cum invitátus fúeris ad núptias, non discúmbas in primo loco, ne forte honorátior te sit invitátus ab illo, et véniens is, qui te et illum vocávit, dicat tibi: Da huic locum: et tunc incípias cum rubóre novíssimum locum tenére. Sed cum vocátus fúeris, vade, recúmbe in novíssimo loco: ut, cum vénerit, qui te invitávit, dicat tibi: Amíce, ascénde supérius. Tunc erit tibi glória coram simul discumbéntibus: quia omnis, qui se exáltat, humiliábitur: et qui se humíliat, exaltábitur.
  }
  \e{%
At that time, when Jesus entered the house of one of the rulers of the Pharisees on the Sabbath to take food, they watched Him. And behold, there was a certain man before Him who had the dropsy. And Jesus asked the lawyers and Pharisees, saying, Is it lawful to cure on the Sabbath? But they remained silent. And He took and healed him and let him go. Then addressing them, He said, Which of you shall have an ass or an ox fall into a pit, and will not immediately draw him up on the Sabbath? And they could give Him no answer to these things. But He also spoke a parable to those invited, observing how they were choosing the first places at table, and He said to them, When you are invited to a wedding feast, do not recline in the first place, lest perhaps one more distinguished than you have been invited by him, and he who invited you and him come and say to you, ‘Make room for this man’; and then you begin with shame to take the last place. But when you are invited go and recline in the last place; that when he who invited you comes in, he may say to you, ‘Friend, go up higher!’ Then you will be honored in the presence of all who are at table with you. For everyone who exalts himself shall be humbled, and he who humbles himself shall be exalted.
  }
}
\newcommand{\offertory}{%
  Dómine, in auxílium meum réspice: confundántur et revereántur, qui quærunt ánimam meam, ut áuferant eam: Dómine, in auxílium meum réspice.
}
\newcommand{\offertoryTranslation}{%
  Deign, O Lord, to rescue me; let all be put to shame and confusion who seek to snatch away my life. Deign, O Lord, to rescue me.
}
\newcommand{\secret}{%
  \l{%
Munda nos, quǽsumus, Dómine, sacrifícii præséntis efféctu: et pérfice miserátus in nobis; ut eius mereámur esse partícipes.
  }
  \e{%
    Cleanse us by this sacrifice, we beseech You, O Lord, and by the workings of Your mercy, make us worthy to receive it.
  }
  \per
}
\newcommand{\communion}{%
  Dómine, memorábor iustítiæ tuæ solíus: Deus, docuísti me a iuventúte mea: et usque in senéctam et sénium, Deus, ne derelínquas me.
}
\newcommand{\communionTranslation}{%
O Lord, I will tell of Your singular justice; O God, You have taught me from my youth; and now that I am old and gray, O God, forsake me not.
}
\newcommand{\postcommunion}{%
  \l{%
Purífica, quǽsumus, Dómine, mentes nostras benígnus, et rénova cœléstibus sacraméntis: ut consequénter et córporum præsens páriter et futúrum capiámus auxílium.
  }
  \e{%
O Lord, we beseech You, graciously cleanse and renew our minds with the heavenly sacrament, so we may thereby also receive bodily help for the present as well as for the future.
  }
  \per
}

% File paths: we don't use symlinks as (a) not all platforms support them, and
% (b) they don't fit nicely with the flow we're using.

\newcommand{\kyriePath}{../Ordinaries/masses/11/kyrie}
\newcommand{\gloriaPath}{../Ordinaries/masses/11/gloria}
\newcommand{\sanctusPath}{../Ordinaries/masses/11/sanctus}
\newcommand{\agnusPath}{../Ordinaries/masses/11/agnus}
\newcommand{\itePath}{../Ordinaries/masses/11/ite}
% 
\newcommand{\creedPath}{../Ordinaries/credo/1/credo}
% 
%%% Local Variables:
%%% mode: latex
%%% TeX-master: "missalette"
%%% End:
