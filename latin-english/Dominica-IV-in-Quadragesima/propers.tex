% This file defines the properties to be inserted into missalette.tex  In this way
% multiple feasts can be typeset very quickly.  missalette.tex should not normally
% need changing.  Note that this is not the most readable way to insert text
% into a LaTeX document, but it is the most powerful: the macros defined here
% are directly executed when building the document.

% For the title page
\newcommand{\feast}{Dominica IV in Quadragesima}
\newcommand{\masstype}{ Missa Cantata%
  % %
}
% 
\newcommand{\introit}{%
Lætáre, Ierúsalem: et convéntum fácite, omnes qui dilígitis eam: gaudéte cum lætítia, qui in tristítia fuístis: ut exsultétis, et satiémini ab ubéribus consolatiónis vestræ.
Lætátus sum in his, quæ dicta sunt mihi: in domum Dómini íbimus.
}
\newcommand{\introitTranslation}{%
Rejoice, O Jerusalem, and come together, all you who love her: rejoice with joy, you who have been in sorrow: that you may exult, and be filled from the breasts of your consolation.
I rejoiced because they said to me, We will go up to the house of the Lord.
}
% 
\newcommand{\collect}{%
  \l{%
Concéde, quǽsumus, omnípotens Deus: ut, qui ex mérito nostræ actiónis afflígimur, tuæ grátiæ consolatióne respirémus.
  }
  \e{%
Grant, we beseech You, almighty God, that we who justly suffer for our sins may find relief in the help of Your grace.
  }
  \perDominum
}
\newcommand{\lesson}{%
  \l{%
Fratres: Scriptum est: Quóniam Abraham duos fílios hábuit: unum de ancílla, et unum de líbera. Sed qui de ancílla, secúndum carnem natus est: qui autem de líbera, per repromissiónem: quæ sunt per allegoríam dicta. Hæc enim sunt duo testaménta. Unum quidem in monte Sina, in servitútem génerans: quæ est Agar: Sina enim mons est in Arábia, qui coniúnctus est ei, quæ nunc est Ierúsalem, et servit cum fíliis suis. Illa autem, quæ sursum est Ierúsalem, líbera est, quæ est mater nostra. Scriptum est enim: Lætáre, stérilis, quæ non paris: erúmpe, et clama, quæ non párturis: quia multi fílii desértæ, magis quam eius, quæ habet virum. Nos autem, fratres, secúndum Isaac promissiónis fílii sumus. Sed quómodo tunc is, qui secúndum carnem natus fúerat, persequebátur eum, qui secúndum spíritum: ita et nunc. Sed quid dicit Scriptúra? Eiice ancíllam et fílium eius: non enim heres erit fílius ancíllæ cum fílio líberæ. Itaque, fratres, non sumus ancíllæ fílii, sed líberæ: qua libertáte Christus nos liberávit.
  }
  \e{%
Brethren: It is written that Abraham had two sons, the one by a slave-girl and the other by a free woman. And the son of the slave-girl was born according to the flesh, but the son of the free woman in virtue of the promise. This said by way of allegory. For these are the two covenants: one indeed from Mount Sinai bringing forth children unto bondage, which is Agar. For Sinai is a mountain in Arabia, which corresponds to the present Jerusalem, and is in slavery with her children. But that Jerusalem which is above is free, which is our mother. For it is written, Rejoice, O barren one, that do not bear; break forth and cry, you that do not travail; for many are the children of the desolate, more than of her that has a husband. Now we, brethren, are the children of the promise, as Isaac was. But as then he who was born according to the flesh, persecuted him who was born according to the spirit, so also it is now. But what does the Scripture say? Cast out the slave-girl and her son, for the son of the slave-girl shall not be heir with the son of the free woman.Therefore, brethren, we are not children of a slave-girl, but of the free woman - in virtue of the freedom wherewith Christ has made us free.
  }
}

% 
\newcommand{\gradual}{%
Lætátus sum in his, quæ dicta sunt mihi: in domum Dómini íbimus.
Fiat pax in virtúte tua: et abundántia in túrribus tuis.
Qui confídunt in Dómino, sicut mons Sion: non commovébitur in ætérnum, qui hábitat in Ierúsalem.
Montes in circúitu eius: et Dóminus in circúitu pópuli sui, ex hoc nunc et usque in sǽculum.
}
\newcommand{\gradualTranslation}{%
I rejoiced because they said to me, We will go up to the house of the Lord.
May peace be within your walls, prosperity in your buildings.
They who trust in the Lord are like Mount Sion, which is immovable; which forever stands.
Mountains are round about Jerusalem; so the Lord is round about His people, both now and forever.
}
% 
% 
\newcommand{\gospel}{%
  \l{%
In illo témpore: Abiit Iesus trans mare Galilǽæ, quod est Tiberíadis: et sequebátur eum multitúdo magna, quia vidébant signa, quæ faciébat super his, qui infirmabántur. Súbiit ergo in montem Iesus: et ibi sedébat cum discípulis suis. Erat autem próximum Pascha, dies festus Iudæórum. Cum sublevásset ergo óculos Iesus et vidísset, quia multitúdo máxima venit ad eum, dixit ad Philíppum: Unde emémus panes, ut mandúcent hi? Hoc autem dicébat tentans eum: ipse enim sciébat, quid esset factúrus. Respóndit ei Philíppus: Ducentórum denariórum panes non suffíciunt eis, ut unusquísque módicum quid accípiat. Dicit ei unus ex discípulis eius, Andréas, frater Simónis Petri: Est puer unus hic, qui habet quinque panes hordeáceos et duos pisces: sed hæc quid sunt inter tantos? Dixit ergo Iesus: Fácite hómines discúmbere. Erat autem fænum multum in loco. Discubuérunt ergo viri, número quasi quinque míllia. Accépit ergo Iesus panes, et cum grátias egísset, distríbuit discumbéntibus: simíliter et ex píscibus, quantum volébant. Ut autem impléti sunt, dixit discípulis suis: Collígite quæ superavérunt fragménta, ne péreant. Collegérunt ergo, et implevérunt duódecim cóphinos fragmentórum ex quinque pánibus hordeáceis, quæ superfuérunt his, qui manducáverant. Illi ergo hómines cum vidíssent, quod Iesus fécerat signum, dicébant: Quia hic est vere Prophéta, qui ventúrus est in mundum. Iesus ergo cum cognovísset, quia ventúri essent, ut ráperent eum et fácerent eum regem, fugit íterum in montem ipse solus.
  }
  \e{%
At that time, Jesus went away to the other side of the sea of Galilee, which is that of Tiberias. And there followed Him a great crowd, because they witnessed the signs He worked on those who were sick. Jesus therefore went up the mountain, and sat there with His disciples. Now the Passover, the feast of the Jews, was near. When, therefore, Jesus had lifted up His eyes and seen that a very great crowd had come to Him, He said to Philip, Whence shall we buy bread that these may eat? But He said this to try him, for He Himself knew what He would do. Philip answered Him, Two hundred denarii worth of bread is not enough for them, that each one may receive a little. One of His disciples, Andrew, the brother of Simon Peter, said to Him, There is a young boy here who has five barley loaves and two fishes; but what are these among so many? Jesus then said, Make the people recline. Now there was much grass in the place. The men therefore reclined, in number about five thousand. Jesus then took the loaves, and when He had given thanks, distributed them to those reclining; and likewise the fishes, as much as they wished. But when they were filled, He said to His disciples, Gather the fragments that are left over, lest they be wasted. They therefore gathered them up; and they filled twelve baskets with the fragments of the five barley loaves left over by those who had eaten. When the people, therefore, had seen the sign which Jesus had worked, they said, This is indeed the Prophet Who is to come into the world. So when Jesus perceived that they would come to take Him by force and make Him king He fled again to the mountain, Himself alone.
  }
}
\newcommand{\offertory}{%
Laudáte Dóminum, quia benígnus est: psállite nómini eius, quóniam suávis est: ómnia, quæcúmque vóluit, fecit in cœlo et in terra.
}
\newcommand{\offertoryTranslation}{%
Praise the Lord, for He is good; sing praise to His Name, for He is sweet; all that He wills He does in heaven and on earth.
}
\newcommand{\secret}{%
  \l{%
Sacrifíciis præséntibus, Dómine, quǽsumus, inténde placátus: ut et devotióni nostræ profíciant et salúti.
  }
  \e{%
Look with favor, we beseech You, O Lord, upon the offerings here before You, that they may be beneficial for our devotion and salvation.
  }
  \perDominum
}
\newcommand{\communion}{%
Ierúsalem, quæ ædificátur ut cívitas, cuius participátio eius in idípsum: illuc enim ascendérunt tribus, tribus Dómini, ad confiténdum nómini tuo, Dómine.
}
\newcommand{\communionTranslation}{%
Jerusalem, built as a city, with compact unity: to it the tribes go up, the tribes of the Lord, to give thanks to Your Name, O Lord.
}
\newcommand{\postcommunion}{%
  \l{%
Da nobis, quǽsumus, miséricors Deus: ut sancta tua, quibus incessánter explémur, sincéris tractémus obséquiis, et fidéli semper mente sumámus.
  }
  \e{%
Grant, we beseech You, merciful God, that we may treat with sincere reverence, and consume with heartfelt faith Your sacrament, which ever fills us to overflowing.
  }
  \perDominum
}

% File paths: we don't use symlinks as (a) not all platforms support them, and
% (b) they don't fit nicely with the flow we're using.

\newcommand{\aspergesPath}{../ToniCommunes/roman/asperges}

\newcommand{\kyriePath}{../Ordinaries/masses/17/kyrie}

\newcommand{\sanctusPath}{../Ordinaries/masses/17/sanctus}

\newcommand{\agnusPath}{../Ordinaries/masses/17/agnus}

\newcommand{\itePath}{../Ordinaries/masses/17/ite}


\newcommand{\creedPath}{../Ordinaries/credo/1/credo}


\newcommand{\amenPath}{../ToniCommunes/roman/amen}

\newcommand{\dominusVobiscumPath}{../ToniCommunes/roman/dominus-vobiscum}

\newcommand{\paxDominiPath}{../ToniCommunes/roman/pax-domini}

\newcommand{\prefacePath}{../ToniCommunes/roman/preface_standard}

\newcommand{\sedLiberaNosPath}{../ToniCommunes/roman/sed-libera-nos}

\newcommand{\sequentiPath}{../ToniCommunes/roman/sequenti}


\newcommand{\marianPath}{../MarianAntiphons/roman/ave-regina-caelorum}
\input{../MarianAntiphons/ave-regina-caelorum_resp}

%%% Local Variables:
%%% mode: latex
%%% TeX-master: "missalette"
%%% End:
