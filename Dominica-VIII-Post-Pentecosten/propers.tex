% This file defines the propert to be inserted into missalette.tex  In this way
% multiple feasts can be typeset very quickly.  missalette.tex should not normally
% need changing.  Note that this is not the most readable way to insert text
% into a LaTeX document, but it is the most powerful: the macros defined here
% are directly excecuted when building the document.

% For the title page
\newcommand{\feast}{Dominica VIII Post Pentecosten}
\newcommand{\masstype}{ Missa Cantata%
  % %
}
% 
\newcommand{\introit}{%
Suscépimus, Deus, misericórdiam tuam in médio templi tui: secúndum nomen tuum, Deus, ita et laus tua in fines terræ: iustítia plena est déxtera tua.
Magnus Dóminus, et laudábilis nimis: in civitáte Dei nostri, in monte sancto eius.
}
\newcommand{\introitTranslation}{%
O God, we ponder Your kindness within Your temple. As Your name, O God, so also Your praise reaches to the ends of the earth. Of justice Your right hand is full.
Great is the Lord and wholly to be praised in the city of our God, His holy mountain.
}
% 
\newcommand{\collect}{%
  \l{%
    Largíre nobis, quǽsumus, Dómine, semper spíritum cogitándi quæ recta sunt, propítius et agéndi: ut, qui sine te esse non póssumus, secúndum te vívere valeámus.
  }
  \e{%
Ever graciously bestow upon us, we beseech You, O Lord, the spirit of thinking and doing what is right, so that we, who cannot exist without You, may have the strength to live in accordance with Your law.
  }
  \per
}
\newcommand{\lesson}{%
  \l{%
Fratres: Debitóres sumus non carni, ut secúndum carnem vivámus. Si enim secúndum carnem vixéritis, moriémini: si autem spíritu facta carnis mortificavéritis, vivétis. Quicúmque enim spíritu Dei aguntur, ii sunt fílii Dei. Non enim accepístis spíritum servitútis íterum in timóre, sed accepístis spíritum adoptiónis filiórum, in quo clamámus: Abba - Pater. - Ipse enim Spíritus testimónium reddit spirítui nostro, quod sumus fílii Dei. Si autem fílii, et herédes: herédes quidem Dei, coherédes autem Christi.
  }
  \e{%
Brethren: We are debtors, not to the flesh, that we should live according to the flesh, for if you live according to the flesh you will die; but if by the spirit you put to death the deeds of the flesh, you will live. For whoever are led by the Spirit of God, they are the sons of God. Now you have not received a spirit of bondage so as to be again in fear, but you have received a spirit of adoption as sons, by virtue of which we cry, Abba! Father! The Spirit Himself gives testimony to our spirit that we are sons of God. But if we are sons, we are heirs also: heirs indeed of God and joint heirs with Christ.
  }
}

% 
\newcommand{\gradual}{%
Esto mihi in Deum protectórem, et in locum refúgii, ut salvum me fácias.
Deus, in te sperávi: Dómine, non confúndar in ætérnum.
Allelúia, allelúia.
Magnus Dóminus, et laudábilis valde, in civitáte Dei nostri, in monte sancto eius. Allelúia.
}
\newcommand{\gradualTranslation}{%
Be my rock of refuge, O God, a stronghold to give me safety.
In You, O God, I take refuge; O Lord, let me never be put to shame. Alleluia, alleluia.
Great is the Lord and wholly to be praised in the city of our God, His holy mountain. Alleluia.
}
% 
% 
\newcommand{\gospel}{%
  \l{%
In illo témpore: Dixit Iesus discípulis suis parábolam hanc: Homo quidam erat dives, qui habébat víllicum: et hic diffamátus est apud illum, quasi dissipásset bona ipsíus. Et vocávit illum et ait illi: Quid hoc audio de te? redde ratiónem villicatiónis tuæ: iam enim non póteris villicáre. Ait autem víllicus intra se: Quid fáciam, quia dóminus meus aufert a me villicatiónem? fódere non váleo, mendicáre erubésco. Scio, quid fáciam, ut, cum amótus fúero a villicatióne, recípiant me in domos suas. Convocátis itaque síngulis debitóribus dómini sui, dicébat primo: Quantum debes dómino meo? At ille dixit: Centum cados ólei. Dixítque illi: Accipe cautiónem tuam: et sede cito, scribe quinquagínta. Deínde álii dixit: Tu vero quantum debes? Qui ait: Centum coros trítici. Ait illi: Accipe lítteras tuas, et scribe octogínta. Et laudávit dóminus víllicum iniquitátis, quia prudénter fecísset: quia fílii huius sǽculi prudentióres fíliis lucis in generatióne sua sunt. Et ego vobis dico: fácite vobis amicos de mammóna iniquitátis: ut, cum defecéritis, recípiant vos in ætérna tabernácula.
  }
  \e{%
At that time, Jesus spoke to His disciples this parable: There was a certain rich man who had a steward, who was reported to him as squandering his possessions. And he called him and said to him, ‘What is this that I hear of you? Make an accounting of your stewardship, for you can be steward no longer.’ And the steward said within himself, ‘What shall I do, seeing that my master is taking away the stewardship from me? To dig I am not able; to beg I am ashamed. I know what I shall do, that when I am removed from my stewardship they may receive me into their houses.’ And he summoned each of his master’s debtors and said to the first, ‘How much do you owe my master?’ And he said, ‘A hundred jars of oil.’ He said to him, ‘Take your bond and sit down at once and write fifty.’ Then he said to another, ‘How much do you owe?’ He said, ‘A hundred kors of wheat.’ He said to him, ‘Take your bond and write eighty.’ And the master commended the unjust steward, in that he had acted prudently; for the children of this world, in relation to their own generation, are more prudent than the children of the light. And I say to you, make friends for yourselves with the mammon of wickedness, so that when you fail they may receive you into the everlasting dwellings.
  }
}
\newcommand{\offertory}{%
Pópulum húmilem salvum fácies, Dómine, et óculos superbórum humiliábis: quóniam quis Deus præter te, Dómine?
}
\newcommand{\offertoryTranslation}{%
Lowly people You save, O Lord, but haughty eyes You bring low; for who is God except You, O Lord?
}
\newcommand{\secret}{%
  \l{%
Súscipe, quǽsumus, Dómine, múnera, quæ tibi de tua largitáte deférimus: ut hæc sacrosáncta mystéria, grátiæ tuæ operánte virtúte, et præséntis vitæ nos conversatióne sanctíficent, et ad gáudia sempitérna perdúcant.
  }
  \e{%
Accept, we beseech You, O Lord, the gifts which we bring to You out of Your own bounty, so that these most holy sacramental rites may, by the power of Your grace, sanctify us in the conduct of our present life, and lead us to everlasting joy.
  }
  \per
}
\newcommand{\communion}{%
Gustáte et vidéte, quóniam suávis est Dóminus: beátus vir, qui sperat in eo.
}
\newcommand{\communionTranslation}{%
Taste and see how good the Lord is; happy the man who takes refuge in Him.
}
\newcommand{\postcommunion}{%
  \l{%
Sit nobis, Dómine, reparátio mentis et córporis cæléste mystérium: ut, cuius exséquimur cultum, sentiámus efféctum.
  }
  \e{%
May the heavenly sacrament, O Lord, renew our minds and bodies, so that we may feel the benefit of the worship we perform.
  }
  \per
}

% File paths: we don't use symlinks as (a) not all platforms support them, and
% (b) they don't fit nicely with the flow we're using.


\newcommand{\kyriePath}{../Ordinaries/masses/11/kyrie}

\newcommand{\gloriaPath}{../Ordinaries/masses/11/gloria}

\newcommand{\sanctusPath}{../Ordinaries/masses/11/sanctus}

\newcommand{\agnusPath}{../Ordinaries/masses/11/agnus}

\newcommand{\itePath}{../Ordinaries/masses/11/ite}


\newcommand{\creedPath}{../Ordinaries/credo/1/credo}


\newcommand{\amenPath}{../ToniCommunes/roman/amen}

\newcommand{\aspergesPath}{../ToniCommunes/roman/asperges}

\newcommand{\dominusVobiscumPath}{../ToniCommunes/roman/dominus-vobiscum}

\newcommand{\paxDominiPath}{../ToniCommunes/roman/pax-domini}

\newcommand{\prefacePath}{../ToniCommunes/roman/preface_standard}

\newcommand{\sedLiberaNosPath}{../ToniCommunes/roman/sed-libera-nos}

\newcommand{\sequentiPath}{../ToniCommunes/roman/sequenti}


\newcommand{\marianPath}{../MarianAntiphons/roman/salve-regina}
\input{../MarianAntiphons/salve-regina_resp}

%%% Local Variables:
%%% mode: latex
%%% TeX-master: "missalette"
%%% End:
